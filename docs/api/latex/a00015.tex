

 layout\+: post title\+: Astrophysics Reactions \hyperlink{a00038}{A\+S\+C\+II} File Format permalink\+: /science/astrophysics/reactions-\/ascii-\/format \subsection*{category\+: science }

Simulations of thermonuclear networks in astrophysics require lists of reactions that describe how the species evolve over time. Fire supports a legacy \hyperlink{a00038}{A\+S\+C\+II} format with a 8-\/line specification for a reaction. Each entry in each line is separated by a space. See \hyperlink{a00678_source}{Reaction.\+h} for full details on each required value.

The first line for each reaction contains\+:


\begin{DoxyItemize}
\item Name/\+Label -\/ A string of the form \char`\"{}he4+he4+he4-\/-\/$>$c12\char`\"{} that describes the reaction.
\item Reaction Group Class -\/ an integer
\item Reaction Group Index -\/ an integer
\item R\+E\+A\+C\+L\+IB Class -\/ an integer
\item Number of reacting species -\/ an integer
\item Number of resulting products -\/ an integer
\item Electron capture flag -\/ a bool
\item Reverse reaction flag -\/ a bool
\item Statistical factor -\/ a double
\item Energy release -\/ a double
\end{DoxyItemize}

The second line contains the seven rate coefficients from the R\+E\+A\+C\+L\+IB library, all doubles. The third line contains the array of atomic numbers for the reactants in this reaction, which are integers. The fourth, fifth, and sixth lines are also integers for the reactant neutron number and product atomic and neutron numbers. Each of these four lines has one integer for each reactant and product in the system, but with no more than four entries per line. (See example below for more details.)

The seventh and eighth lines contain quantities that are used for {\itshape Partial Equilibrium} approximations. Each line contains up to three integers.


\begin{DoxyCode}
1 he4+he4+he4-->c12 3 0 8 3 1 0 0 0.16666667 7.27500
2 -24.99350000 -4.29702000 -6.69304000 15.59030000 -1.57387000 0.17058800 -9.02800000
3 2 2 2 // Since numReactants = 3 and he4 has Z=2, there are three values on this line equal to 2.
4 2 2 2 // Same, but for he4's neutron number
5 6 // There is only one product and Z=6 for C12.
6 6 // N=6 for C12
7 0 0 0 
8 1 
9 ne20-->he4+o16 2 3 2 1 2 0 1 1.00000000 -4.73400
10 109.31000000 -72.75840000 293.66400000 -384.97400000 20.23800000 -1.00379000 201.19300000
11 10 // Z for ne20
12 10 // N for ne20
13 2 8 // Z for he4 and o16
14 2 8 // N for he4 and o16
15 3 
16 0 2 
\end{DoxyCode}
 