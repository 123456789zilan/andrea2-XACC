

 layout\+: post title\+: Nuclear Reactions in Astrophysical Systems permalink\+: /science/astrophysics/reactions \subsection*{category\+: science }

\section*{Overview}

The astrophysics module of Fire contains a set of classes for managing thermonuclear reactions in astrophysical problems. The Reaction\+Network class is responsible for organizing the species and rates that drive the evolution of the network.

\section*{Configuring the Network}

\section*{Reaction Rates}

Reaction rates are computed using parameterized rates from the R\+E\+A\+C\+L\+IB rate library. The R\+E\+A\+C\+L\+IB rates are parameterized by seven coefficients and can be computed as

The rate is computed as

\$\$ R = p\+\_\+s$\ast$ R\+\_\+k \$\$

where \$\$p\+\_\+s\$\$ is the prefactor (based on the statistical prefactor) and

\$\$ R\+\_\+k = (p\+\_\+1 + \{p\+\_\+2\}\{T\+\_\+9\} + \{p\+\_\+3\}\{T\+\_\+9$^\wedge$\{1/3\}\} + p\+\_\+\{4\}T\+\_\+9$^\wedge$\{1/3\}
\begin{DoxyItemize}
\item p\+\_\+\{5\}T\+\_\+9 + p\+\_\+\{6\}T\+\_\+9$^\wedge$\{5/3\} + p\+\_\+\{7\} T\+\_\+9) \$\$
\end{DoxyItemize}

\$\$\+T\+\_\+9\$\$ is the the temperature in units of \$\$10$^\wedge$9\$\$ Kelvin. The prefactor is given by

\$\$ p\+\_\+s = s$^\wedge$\{(n\+\_\+R -\/1)\}. \$\$

where \$\$s\$\$ is the statistical factor (an input from R\+E\+A\+C\+L\+IB), \$\$\$\$ is the density in units of \$\$10$^\wedge$8 \{g\}\{m$^\wedge$3\}\$\$ and \$\$n\+\_\+R\$\$ is the number of reactants in the reaction (species on the left hand side that start the reaction).

In general k may be greater than 1 in the summation for the rate, but in this work k = 1 and \$\$R = R\+\_\+k\$\$.

More information on how this library is parsed is available in the documentation for \href{{{ site.baseurl }}}{\tt the Reaction A\+S\+C\+II format}.

\subsubsection*{Some Design Problems}

Some of the code in the astrophysics module was adapted from an older code called the Fast Explicit Reaction Network (F\+E\+RN) solver. F\+E\+RN is a research code that was developed by students and others at the University of Tennessee to explore issues around efficiently solving reaction networks using advanced methods and hardware.

F\+E\+RN has a very low-\/level design with significant flaws. The routines in F\+E\+RN that initialize the network pack everything into arrays with little regard for treating the memory right, among other problems. When this code was ported to Fire, the memory errors were fixed, but the larger considerations for a cleaner design were not addressed. These problems are discussed in-\/line in the classes themselves and will be addressed in a future release.

\section*{References}

\char`\"{}\+Stars and Stellar Processes\char`\"{}, Mike Guidry, to be published Cambridge University Press. 