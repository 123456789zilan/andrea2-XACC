

 layout\+: post title\+: Parsers permalink\+: /design/parsers \subsection*{category\+: design }

Parsers in Fire implement the I\+Parser interface, which can either be implemented through direct interface realization or by using template specialization for local parsers when appropriate.

\subsection*{Local Parsers}

Local Parsers in Fire represent parsers for resources that are on the local machine (i.\+e. -\/ localhost). These can be streams or files, depending on the implementation. Local Parsers implement I\+Local\+Parser or specialize the Local\+Parser$<$T$>$ template.

\subsubsection*{Templated Local Parsers}

Specializations of Local\+Parser$<$T$>$ should be created in the \char`\"{}fire\char`\"{} namespace. See \hyperlink{a00332_source}{astrophysics/\+Species\+Local\+Parser.\+h} for an example.

This is required by the C++ standard. If you do not create your specialization in the \char`\"{}fire\char`\"{} namespace, you will get an error like the following


\begin{DoxyCode}
error: specialization of ‘template<class T> class fire::LocalParser’ in different namespace [-fpermissive]
 class LocalParser<std::vector<Species>> \{
\end{DoxyCode}


\subsubsection*{Delimited Text Parsers}

One special class of parsers are parsers for delimited text, which is dense text with entries of the same type separated by a common {\itshape delimiter,} like a comma, space, or other character. Fire has a special subclass of Local\+Parser, called the Delimited\+Text\+Parser, that should be used for this type of text.

\subsection*{Parsing Simplified}

The simplest way to use a local parser in Fire is to use the parse$<$$>$() template function. If you implemented Local\+Parser$<$T$>$\+::parse() as described above, the you are most likely parsing your file like this


\begin{DoxyCode}
LocalParser<T> parser;
parser.setSource(fileName);
parser.parse();
shared\_ptr<T> myData = parser.getData();
\end{DoxyCode}


or perhaps by using a builder like this


\begin{DoxyCode}
\textcolor{keyword}{auto} parser = build<LocalParser<vector<T>>, \textcolor{keyword}{const} \textcolor{keywordtype}{string} &>(fileName);
parser.parse();
\textcolor{keyword}{auto} myData = speciesParser.getData();
\end{DoxyCode}


Both of those are great, but you can get the same function in Fire using parse$<$T$>$() as follows


\begin{DoxyCode}
\textcolor{keyword}{auto} myData = parse<T>(filename);
\end{DoxyCode}


In this case, the parse$<$T$>$() function will automatically create the parser and return your data using a Local\+Parser$<$T$>$ if it is available. parse$<$T$>$() is specifically designed for returning vectors of type T from local files. It isn\textquotesingle{}t a substitute for complex parsers, but it saves a lot of effort for regular, block-\/structured data. 